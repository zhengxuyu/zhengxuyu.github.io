\documentclass[10pt, letterpaper]{article}

% Packages:
\usepackage[
    ignoreheadfoot, % set margins without considering header and footer
    top=2 cm, % seperation between body and page edge from the top
    bottom=2 cm, % seperation between body and page edge from the bottom
    left=2 cm, % seperation between body and page edge from the left
    right=2 cm, % seperation between body and page edge from the right
    footskip=1.0 cm, % seperation between body and footer
    % showframe % for debugging 
]{geometry} % for adjusting page geometry
\usepackage{titlesec} % for customizing section titles
\usepackage{tabularx} % for making tables with fixed width columns
\usepackage{array} % tabularx requires this
\usepackage[dvipsnames]{xcolor} % for coloring text
\definecolor{primaryColor}{RGB}{0, 79, 144} % define primary color
\usepackage{enumitem} % for customizing lists
\usepackage{fontawesome5} % for using icons
\usepackage{amsmath} % for math
\usepackage[
    pdftitle={余正旭's CV},
    pdfauthor={余正旭},
    pdfcreator={LaTeX with RenderCV},
    colorlinks=true,
    urlcolor=primaryColor
]{hyperref} % for links, metadata and bookmarks
\usepackage[pscoord]{eso-pic} % for floating text on the page
\usepackage{calc} % for calculating lengths
\usepackage{bookmark} % for bookmarks
\usepackage{lastpage} % for getting the total number of pages
\usepackage{changepage} % for one column entries (adjustwidth environment)
\usepackage{paracol} % for two and three column entries
\usepackage{ifthen} % for conditional statements
\usepackage{needspace} % for avoiding page brake right after the section title
\usepackage{iftex} % check if engine is pdflatex, xetex or luatex

% Ensure that generate pdf is machine readable/ATS parsable:
\ifPDFTeX
    \input{glyphtounicode}
    \pdfgentounicode=1
    \usepackage[T1]{fontenc}
    \usepackage[utf8]{inputenc}
    \usepackage{lmodern}
\fi

\usepackage[default, type1]{sourcesanspro} 

% Some settings:
\AtBeginEnvironment{adjustwidth}{\partopsep0pt} % remove space before adjustwidth environment
\pagestyle{empty} % no header or footer
\setcounter{secnumdepth}{0} % no section numbering
\setlength{\parindent}{0pt} % no indentation
\setlength{\topskip}{0pt} % no top skip
\setlength{\columnsep}{0.15cm} % set column seperation
\makeatletter
\let\ps@customFooterStyle\ps@plain % Copy the plain style to customFooterStyle
\patchcmd{\ps@customFooterStyle}{\thepage}{
    \color{gray}\textit{\small 余正旭 - Page \thepage{} of \pageref*{LastPage}}
}{}{} % replace number by desired string
\makeatother
\pagestyle{customFooterStyle}

\titleformat{\section}{\needspace{4\baselineskip}\bfseries\large}{}{0pt}{}[\vspace{1pt}\titlerule]

\titlespacing{\section}{
    % left space:
    -1pt
}{
    % top space:
    0.3 cm
}{
    % bottom space:
    0.2 cm
} % section title spacing

\renewcommand\labelitemi{$\vcenter{\hbox{\small$\bullet$}}$} % custom bullet points
\newenvironment{highlights}{
    \begin{itemize}[
        topsep=0.10 cm,
        parsep=0.10 cm,
        partopsep=0pt,
        itemsep=0pt,
        leftmargin=0.4 cm + 10pt
    ]
}{
    \end{itemize}
} % new environment for highlights


\newenvironment{highlightsforbulletentries}{
    \begin{itemize}[
        topsep=0.10 cm,
        parsep=0.10 cm,
        partopsep=0pt,
        itemsep=0pt,
        leftmargin=10pt
    ]
}{
    \end{itemize}
} % new environment for highlights for bullet entries

\newenvironment{onecolentry}{
    \begin{adjustwidth}{
        0.2 cm + 0.00001 cm
    }{
        0.2 cm + 0.00001 cm
    }
}{
    \end{adjustwidth}
} % new environment for one column entries

\newenvironment{twocolentry}[2][]{
    \onecolentry
    \def\secondColumn{#2}
    \setcolumnwidth{\fill, 4.5 cm}
    \begin{paracol}{2}
}{
    \switchcolumn \raggedleft \secondColumn
    \end{paracol}
    \endonecolentry
} % new environment for two column entries

\newenvironment{threecolentry}[3][]{
    \onecolentry
    \def\thirdColumn{#3}
    \setcolumnwidth{, \fill, 4.5 cm}
    \begin{paracol}{3}
    {\raggedright #2} \switchcolumn
}{
    \switchcolumn \raggedleft \thirdColumn
    \end{paracol}
    \endonecolentry
} % new environment for three column entries

\newenvironment{header}{
    \setlength{\topsep}{0pt}\par\kern\topsep\centering\linespread{1.5}
}{
    \par\kern\topsep
} % new environment for the header

\newcommand{\placelastupdatedtext}{% \placetextbox{<horizontal pos>}{<vertical pos>}{<stuff>}
  \AddToShipoutPictureFG*{% Add <stuff> to current page foreground
    \put(
        \LenToUnit{\paperwidth-2 cm-0.2 cm+0.05cm},
        \LenToUnit{\paperheight-1.0 cm}
    ){\vtop{{\null}\makebox[0pt][c]{
        \small\color{gray}\textit{Last updated in December 2024}\hspace{\widthof{Last updated in December 2024}}
    }}}%
  }%
}%

% save the original href command in a new command:
\let\hrefWithoutArrow\href

% new command for external links:
\renewcommand{\href}[2]{\hrefWithoutArrow{#1}{\ifthenelse{\equal{#2}{}}{ }{#2 }\raisebox{.15ex}{\footnotesize \faExternalLink*}}}


\begin{document}
    \newcommand{\AND}{\unskip
        \cleaders\copy\ANDbox\hskip\wd\ANDbox
        \ignorespaces
    }
    \newsavebox\ANDbox
    \sbox\ANDbox{}

    \placelastupdatedtext
    \begin{header}
        \fontsize{30 pt}{30 pt}\selectfont 余正旭

        \vspace{0.3 cm}

        \normalsize
        \mbox{\hrefWithoutArrow{mailto:yuzxfred@gmail.com}{{\footnotesize\faEnvelope[regular]}\hspace*{0.13cm}yuzxfred@gmail.com}}%
        \kern 0.25 cm%
        \AND%
        \kern 0.25 cm%
        \mbox{\hrefWithoutArrow{tel:+44-7852-446689}{{\footnotesize\faPhone*}\hspace*{0.13cm}07852 446689}}%
        \kern 0.25 cm%
        \AND%
        \kern 0.25 cm%
        \mbox{\hrefWithoutArrow{https://zhengxuyu.github.io/}{{\footnotesize\faLink}\hspace*{0.13cm}zhengxuyu.github.io}}%
        \kern 0.25 cm%
        \AND%
        \kern 0.25 cm%
        \mbox{\hrefWithoutArrow{https://linkedin.com/in/https://www.linkedin.com/in/yuzhengxu}{{\footnotesize\faLinkedinIn}\hspace*{0.13cm}https://www.linkedin.com/in/yuzhengxu}}%
        \kern 0.25 cm%
        \AND%
        \kern 0.25 cm%
        \mbox{\hrefWithoutArrow{https://github.com/https://github.com/zhengxuyu}{{\footnotesize\faGithub}\hspace*{0.13cm}https://github.com/zhengxuyu}}%
    \end{header}

    \vspace{0.3 cm - 0.3 cm}


    \section{Intro}



        
        \begin{onecolentry}
            余正旭现为阿里巴巴集团的高级研究员。他获得了浙江大学博士学位,导师为蔡登和何晓飞教授。在此之前,他于英国萨里大学获得硕士学位,导师为H. Lilian Tang教授。
        \end{onecolentry}

        \vspace{0.2 cm}

        \begin{onecolentry}
            他的研究兴趣包括强化学习、机器学习和通用人工智能,特别关注AI推理和决策模型、AI决策与人类认知的对齐以及动态系统中的随机建模。
        \end{onecolentry}

        \vspace{0.2 cm}

        \begin{onecolentry}
            他的当前研究目标是开发能够以类似人类方式进行推理、学习和决策的通用人工智能模型。
        \end{onecolentry}

        \vspace{0.2 cm}

        \begin{onecolentry}
            余正旭已在人工智能领域的顶级国际会议和期刊(如IJCAI、AAAI、ECCV等)发表了11篇研究论文。
        \end{onecolentry}


    
    \section{Experience}



        
        \begin{twocolentry}{
            Apr 2021 – present
        }
            \textbf{高级算法研究员}, 阿里巴巴集团 -- 中国杭州\end{twocolentry}

        \vspace{0.10 cm}
        \begin{onecolentry}
            \begin{highlights}
                \item 开发了基于强化学习的算法,用于大语言模型(LLM)的后训练范式,重点研究推理和对齐任务。提出了一种新的自博弈方法及多种训练算法,提升了模型在多个基准测试中的推理能力。
                \item 将后训练的LLM模型应用于真实场景,如自动资源分配和设备操作。
                \item 开发了用于时序动态系统(如城市交通系统)的序列推理和预测方法。
                \item 带领跨职能团队交付AI解决方案,并指导研究实习生和初级研究员。
            \end{highlights}
        \end{onecolentry}


        \vspace{0.2 cm}

        \begin{twocolentry}{
            Jan 2018 – Apr 2021
        }
            \textbf{研究实习生}, 达摩院,阿里巴巴集团 -- 中国杭州\end{twocolentry}

        \vspace{0.10 cm}
        \begin{onecolentry}
            \begin{highlights}
                \item 提出多智能体强化学习方法,促进协作与竞争场景下的智能体协调。
                \item 提出多种优化方法,提高深度神经网络在计算机视觉任务中的泛化能力。
                \item 提出基于生成对抗网络(GAN)的数据生成模型,用于增强计算机视觉任务的训练数据。
                \item 提出多种深度图神经网络(GNN)模型,用于动态系统中的随机建模任务。
            \end{highlights}
        \end{onecolentry}



    
    \section{Education}



        
        \begin{twocolentry}{
            Sept 2017 – Mar 2021
        }
            \textbf{浙江大学}, 博士 in 计算机科学\end{twocolentry}

        \vspace{0.10 cm}
        \begin{onecolentry}
            \begin{highlights}
                \item \textbf{研究方向}: 机器学习、计算机视觉、生成模型、数据挖掘
            \end{highlights}
        \end{onecolentry}


        \vspace{0.2 cm}

        \begin{twocolentry}{
            Sept 2015 – Nov 2016
        }
            \textbf{萨里大学}, 硕士 in 信息系统\end{twocolentry}

        \vspace{0.10 cm}
        \begin{onecolentry}
            \begin{highlights}
                \item \textbf{研究方向}: 机器学习、计算机视觉、数据挖掘
            \end{highlights}
        \end{onecolentry}


        \vspace{0.2 cm}

        \begin{twocolentry}{
            Sept 2011 – June 2015
        }
            \textbf{吉林大学}, 学士 in 通信工程\end{twocolentry}




    
    \section{Technologies}



        
        \begin{onecolentry}
            \textbf{编程语言与技术:} Python, PyTorch, Pandas, LangChain, vllm, ray, deepspeed
        \end{onecolentry}


    
    \section{Publications}



        
        \begin{samepage}
            \begin{twocolentry}{
                2023
            }
                \textbf{基于时间分配的强化学习优化交通效率}
            \end{twocolentry}

            \vspace{0.10 cm}
            
            \begin{onecolentry}
                \mbox{Cao, X.}, \mbox{Jin, Z.}, \mbox{\textbf{Yu, Z.}}, \mbox{Hua, X.}, \mbox{Hu, Y.}, \mbox{Qian Wei.}, \mbox{Zhu K.}, \mbox{Cai D.}, \mbox{He, X.}

                \vspace{0.10 cm}
                
        \href{https://doi.org/10.1007/s13042-023-01838-1}{10.1007/s13042-023-01838-1}
         (机器学习与网络学报)\end{onecolentry}
        \end{samepage}

        \vspace{0.2 cm}

        \begin{samepage}
            \begin{twocolentry}{
                2022
            }
                \textbf{渐进迁移学习}
            \end{twocolentry}

            \vspace{0.10 cm}
            
            \begin{onecolentry}
                \mbox{\textbf{Yu, Z.}}, \mbox{Jin, Z.}, \mbox{Wei, L.}, \mbox{Huang, J.}, \mbox{Cai, D.}, \mbox{He, X.}, \mbox{Hua, X.S.}

                \vspace{0.10 cm}
                
        \href{https://doi.org/10.1109/TIP.2022.3141258}{10.1109/TIP.2022.3141258}
         (IEEE图像处理期刊(TIP))\end{onecolentry}
        \end{samepage}


    

\end{document}